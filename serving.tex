% Instructions for Serving Low Mass
% Created: August 14, 2025
% Edited: August 26, 2025
\documentclass[12pt]{article}

% packages
\usepackage[top=2.5cm,left=2.5cm,right=2.5cm,bottom=2.5cm,marginparwidth=1in,marginparsep=2ex]{geometry} % page layout
\usepackage[utf8]{inputenc} % input encodings into TeX-speak
\usepackage[letterspace=250]{microtype} % typographic precision
\usepackage{babel} % international typographical support; [other language,english]
\usepackage{graphicx}
\usepackage{enumitem} % more control over list environments
\usepackage{calc} % calculates lengths and counters; used primarily for calculating environment label lengths
\usepackage{soul} % letter spacing, underlining, strikeouts
\usepackage{paracol} % multiple texts in parallel columns
\usepackage{multicol} % one text in multiple columns
\usepackage{booktabs} % table options
\usepackage{afterpage}
\usepackage{ebgaramond}

% headers
\newcommand{\headerone}[2]{\begin{center}\textsc{\Huge\lsstyle #1}\paragraphspace\Large\scshape#2\end{center}\paragraphspace} % main title
\newcommand{\headertwo}[1]{\begin{center}\textsc{\LARGE #1}\end{center}\quotespace} % highest section title
\newcommand{\headerthree}[1]{\textup{\Large #1}\paragraphspace}
\newcommand{\paragraphheader}[1]{\textbf{\large #1}} % new segment or thought within a section
\newcommand{\subtitle}

% styling
\newcommand{\booktitle}{\textit} % citation of works in italics
\newcommand{\latin}{\textit} % italicize latin words
\newcommand{\margintext}[1]{\textit{\small #1}} % margin notes
\newcommand{\principle}{\ul} %for principles, definitions, and components of the aforesaid
\newcommand{\quoting}{\textit}
%\newcommand{\repeatafter}{%
  %\thispagestyle{empty}%
  %\addtocounter{page}{-1}%
  %\null\newpage
  %\afterpage{\repeatafter}%
%}

% lists
\newcommand{\bnotes}{\begin{itemize}}
\newcommand{\enotes}{\end{itemize}}

\newcommand{\blist}{\begin{enumerate}}
\newcommand{\elist}{\end{enumerate}}

\newcommand{\bdefs}{\begin{description}}
\newcommand{\edefs}{\end{description}}

\newcommand{\pcl}{\begin{paracol}}
\newcommand{\ecl}{\end{paracol}}
\newcommand{\sw}{\switchcolumn}
\newcommand{\swc}{\switchcolumn*}

% list settings
%\setlist[description]{font=\normalfont,align=left}
%[labelwidth=\widthof{}]

% skips-horizontal
\newcommand{\smallwidth}{\hskip 1em}
\newcommand{\medwidth}{\hskip 3pt}
\newcommand{\largewidth}{\hskip 12pt}

% skips-vertical
\newcommand{\linespace}{\vskip 4pt}
\newcommand{\paragraphspace}{\vskip 6pt}
\newcommand{\segmentspace}{\vskip 12pt}
\newcommand{\quotespace}{\vskip 16pt} % quote lead before/after
\newcommand{\sectionspace}{\vskip 24pt}

% page spacings; these are declarations
\parindent=0pt
\newcommand{\parset}{\parindent=1em}
\newcommand{\firstpar}{\noindent}
\renewcommand{\baselinestretch}{1.2}
\emergencystretch=3em
\columnsep=1em
%\reversemarginpar

%%%%%%%%%%
\begin{document}

%%
%% Title page
\thispagestyle{empty}

\headerone{Serving Guide}{Traditional Latin Mass}

\vfill
\begin{center}
\includegraphics[width=\linewidth]{latin-mass}
\end{center}

\vfill

\begin{center}

%\begin{quote}

  \em
  Know that nothing in the world is so precious, so useful, so noble, so sublime as the Holy Mass.

  \quotespace

  St. Leonard of Port Maurice

%\end{quote}

\end{center}

\vfill

%% Blank page = back side of title

%\afterpage{\repeatafter}
\newpage
\thispagestyle{empty}
\null

%%
%% Principles page

\newpage
\thispagestyle{empty}

\headerthree{Liturgical Principles}


Division of Mass
\blist

  \item Mass of the Catechumens -- Preparation for the most holy act of religion.

    \blist

      \item[1.] Prayer -- The beginning of turning to God.

      \item[2.] Instruction -- Hear God's word.

    \elist

  \item Mass of the Faithful -- The sacrifice of the Eucharist.

    \blist

      \item[1.] Sacrifice -- Offer Jesus Christ to God the Father.

      \item[2.] Communion -- Union with the Lamb of God.

    \elist

\elist

Motions and Actions

\begin{quote}

  \begin{center}\em

  The movements of a server at the altar should be grave, reverent, and as noiseless as possible.

  Fr. J. O'Connell

  \end{center}

\end{quote}

\segmentspace

\bnotes

  \item Hands should be in the ``prayer position'' by default. Hands joined and in front of the breast.
  \item If one hand is being used, put the idle hand flat on the chest.
  \item Avoid turning your back to the Blessed Sacrament or the cross.
  \item The default position is kneeling.
  \item Generally, one should kneel on the opposite side of the Missal.
  \item Match the vocal volume of the priest.
  \item Responses must be precise and clear.
  \item The bell should be rung moderately.
  \item Genuflecting is a continuous action. There is no pause when the knee touches the ground.
  \item Actions should be precise and restrained. Do not add gestures.

\enotes

% Blank page = back of principles page

%\afterpage{\repeatafter}
\newpage
\thispagestyle{empty}
\null


%%
%% Mass of the Catechumens

\newpage
\setcounter{page}{3}

\headertwo{Mass of the Catechumens}

\begin{center}
  \vskip -0.4 in
\rule{3.5 in}{0.4 pt}
\end{center}

\headerthree{Prayer}

Prayers at the Foot of the Altar

\bnotes

  \item Server kneels to the left of the priest, slightly behind him.
  \item Alternate responses to Psalm 42 and following prayers, as in the Response Card.
  \item When the priest says \latin{Oremus}, or slightly before, lift up his alb and cassock before his feet to give him a clear path up the steps.
  \item Rise, go to the left of the altar steps (Gospel side), kneel on the lowest step.

\enotes

Introit, Kyrie, Gloria

\bnotes

  \item Make the sign of the cross at the beginning of the Introit.
  \item Alternate \latin{Kyrie eleison} and \latin{Christe eleison} with the priest, according to the Response Card.
  \item During \latin{Gloria in excelsis}, bow head along with the priest.
  \item Make the sign of the cross at the end, \latin{Cum Sancto Spiritu...}
  \item Make the proper responses for the Collects, as in the Response Card.

\enotes

\headerthree{Instruction}

Epistle

\bnotes

  \item At the beginning of the Epistle, the priest will place his hands on the side of the Missal.
  \item When he finishes the Epistle, he will place his left hand on the altar to signal the Epistle is finished.
  \item Respond \latin{Deo gratias}.

\enotes

Gospel

\bnotes

  \item Rise, go to the center, genuflect, walk around the steps to the Epistle corner, where the priest is reading the Gradual and Alleluia verses.
  \item (On some feast days, there is a Sequence. Do not rise and move until the Sequence is almost over.)
  \item When the priest moves toward the center, ascend the altar steps and take the Missal-stand in both hands.
  \item Go down the steps diagonally toward the center.
  \item Genuflect in the center, and go up the steps to the left and place the Missal-stand at an angle in the Gospel corner.
  \item Go to the lowest step and stand so that you will be facing the priest.
  \item Make the responses and the three small signs of the cross.
  \item Wait for the name of Jesus, bow your head, then go down the last step.
  \item (If the Holy Name does not come, leave after several seconds. Do not bow your head.)
  \item Walk around the steps to the opposite corner, genuflecting at the center.
  \item Turn at an angle to face in the same way as the priest.
  \item When he stops reading the Gospel, say \latin{Laus tibi, Christe.}
  \item Face forward and kneel down.

\enotes


Sermon
\bnotes

  \item On Sundays, there will be a sermon.
  \item By exception, remain standing after the Gospel.
  \item The priest will take off his Maniple, then descend down the steps.
  \item Kneel with the priest and pray.
  \item Having prayed, rise, genuflect, and go to your seat.
  \item When the sermon is done, go back to the center.
  \item Genuflect, then lift the priest's alb and cassock for him to rise up the steps.
  \item Kneel on the lowest step.

\enotes


Creed
\bnotes

  \item On Sundays and feast days, the Nicene Creed will be recited.
  \item Stay kneeling for the Creed.
  \item Bow your head when the priest does.
  \item The priest will slowly begin genuflecting at \latin{Et incarnatus est...}
  \item Slowly bow over as he descends. Stop at the profound bow position. Then, rise slowly as he does.
  \item Make the sign of the cross with the priest at \latin{Et vitam venturi...}

\enotes


%%
%% Mass of the Faithful

\newpage

\headertwo{Mass of the Faithful}

\begin{center}
  \vskip -0.35 in
\rule{2.85 in}{0.4 pt}
\end{center}

\headerthree{Sacrifice}

Offertory
\bnotes

  \item Respond to priest and bow head at \latin{Oremus}.
  \item When he finishes reading the Offertory verse, rise and go to the credence table.
  \item Obtain the cruets of wine and water. Remove the stoppers.
  \item Stand near the altar, with the altar table in front of you.
  \item Bow as the priest approaches and kiss the cruets.
  \item Hand the wine when he reaches for it.
  \item Take back the wine and hold out the water for him to bless it. Then, give him the water cruet.
  \item Receive back the water cruet.
  \item Kiss both cruets, bow, and return to the credence table.
  \item Place both cruets down, and take the water dish and finger towel in your left hand.
  \item Take the water cruet in your right hand.
  \item Return to the altar. This time face the open space, where the priest is standing.
  \item Bow as he approaches, and pour a little water over his fingers until he signals to stop.
  \item Present the finger towel or simply let him take it. Present your arm to receive the towel back.
  \item Bow, return to the credence table, and put everything back in place.
  \item Return to your place on the lowest step.
  \item The priest will shortly turn around and say \latin{Orate, fratres.}
  \item Wait until he completes turning, bow down moderately, and respond \latin{Suscipiat...}

\enotes


Preface
\bnotes

  \item When the priest is done silently reading the prayers, he will say out loud, \latin{Per omnia...}
  \item Make the correct responses as indicated in the Response Card.
  \item When he bows and says \latin{Sanctus} for the first time, ring the bell three times.
  \item If you can, make the sign of the cross at \latin{Benedictus} with the priest.

\enotes


Canon
\bnotes

  \item The priest will extend his hands at the prayer called \latin{Hanc Igitur}.
  \item Ring the bell one time.
  \item Then, ascend to the top platform.
  \item When the priest bends over to say the words of consecration, bow down moderately.
  \item Ring the bell when the priest genuflects.
  \item Lift up his chasuble slightly, then ring the bell as he raises the Sacred Host.
  \item Ring the bell a third time when he genuflects.
  \item Bow again when he says the words of consecration over the chalice.
  \item Ring the bell when he genuflects, raise the chasuble again and ring the bell, then ring the bell a third time when he genuflects.
  \item After the consecration, descend to the floor, genuflect, and return to your place on the bottom step.

\enotes


Post-Canon Prayers
\bnotes

  \item Remain in silence until the priest says \latin{Per omnia...}
  \item Respond correctly for this prayer and the next, as in the Response Card.
  \item When the priest says \latin{Agnus Dei}, bow moderately and strike your breast.
  \item The priest will take the Host, bend over, and say \latin{Domine, non sum dignus...}
  \item Ring the bell each time he says this, for three times total.
  \item Bow moderately when he consumes the Host.

\enotes


\headerthree{Communion}

Communion
\bnotes

  \item The priest will take off the pall over the chalice.
  \item Stand up and retrieve the communion paten from the credence table.
  \item Go to the steps, genuflect, and ascend.
  \item Kneel on the top step if you will take communion.
    \bnotes
      \item[•]If you will not communicate, kneel on the bottom step.
    \enotes
  \item When the priest drinks the Precious Blood, bow moderately and recite the \latin{Confiteor} on behalf of the people.
  \item Respond as in the first \latin{Confiteor}.
  \item If you are going to receive, strike your breast at \latin{Domine, non sum dignus...}
  \item Walk with the priest to the communion rail.
  \item Carefully hold the paten under the chin of each person.
    \bnotes
      \item[•]But, make sure to keep moving so that you allow the priest to move fluidly.
    \enotes
  \item After the last person, hand the communion paten back to the priest.
  \item Accompany him to the altar steps. Lift his alb and cassock as he ascends the steps.
  \item Stand at the steps. Genuflect when the priest does.
  \item Then, go to the Epistle side and kneel on the side of the steps.

\enotes


Ablutions
\bnotes

  \item When the priest closes the tabernacle, rise and go to the credence table.
  \item Retrieve the cruets again and wait along the Epistle side.
  \item The priest will have his chalice tilted. Go up and pour wine carefully into the chalice.
  \item When he signals to stop, go back down the steps.
  \item When the priest comes to the Epistle side, carefully pour wine and then water over his fingers into the chalice until he signals to stop.
  \item Return the cruets to the credence table.
  \item Take the communion paten from the altar and put that back on the credence table.
  \item Go the long way to the Gospel side, genuflecting at the middle.
  \item Take the bookstand and go down diagonally to the center. Genuflect.
  \item Ascend the steps diagonally and place the bookstand parallel to the edge.
  \item Descend the stairs on the side. Go to the Gospel side the long way and kneel on the lowest step.
  \item Make all the responses to the following prayers.

\enotes


Conclusion
\bnotes

  \item After the blessing, the priest will say the Last Gospel.
    \bnotes
      \item[•]If there is a proper Last Gospel, the priest will leave the Missal open after the last prayers.
      \item[•]Bring back the bookstand to the Gospel side the same way you did after the Ablutions.
      \item[•]Except, wait for the blessing in the center of the bottom step.
    \enotes
  \item After the priest begins the Last Gospel, go to the Epistle side, genuflecting at the middle.
  \item Kneel on the step with the priest for the Leonine Prayers.
  \item When the prayers are done, lift the priest's alb and cassock as he goes up the steps to get the chalice.
  \item After he goes up the steps, get his biretta.
  \item Kiss the biretta, then the priest's hand.
  \item Return to the sacristy and receive the blessing.
  \item Either prepare for the next Mass or, if that is the last Mass at the altar, put everything back in order.

\enotes

%%%%%%%%%%
\end{document}

